\noindent Currently, vaccination is the unique strategy to eradicate COVID-19. However, its competence is restricted to several factors such as its distribution, stocks, politics, vaccination efforts among others.
 A good distribution or application strategy of these drugs is imperative to manage the available resources, especially in developing countries. In order to design optimal vaccination strategies, a mathematical model has been proposed to describe the transmission dynamics of COVID-19, which also includes vaccination dynamics. Our aim is to provide vaccination policies that minimize reported deaths and optimize resource consumption.
 
Similar to other articles with the same approach, our model presents a reproductive vaccination number $(\mathcal{R}_{v})$ that depends on $\mathcal{R}_{0}$ \cite{Alexander2004}. Here, we can see the importance of the parameters related to vaccination dynamics to reduce $\mathcal{R}_{v}$. However, according to our simulation parameters setup, to obtain $\mathcal{R}_{v} < 1$ we need impractical vaccination rates. This last can be improved when implementing NPIs, which can be modeled by transmission contact rates reduction.

To show our scenario of optimal control strategies, we consider Mexico City as the study scenario. Using the data reported by the onset of symptoms, some parameters were estimated, and others from the literature were set. Our results capture the restriction of health care centers that is, do not overcome the available number of beds. Given a target coverage at final time $T$, we illustrate that an optimal vaccination strategy reduces the number of fatalities compared to the application of the constant vaccination one. 

The present work can be extended in different fronts such as i) the evaluation of the inclusion of NPIs at the same time vaccination implementation takes place; ii) the consequences of applying the vaccination strategies during different phases of an outbreak; iii) the effects of using double vaccination doses; iv) the application of different vaccines to distinct groups of individuals, v) vaccine application with knowledge of the percentage of the population that has resulted positive in a seroprevalence study and vi) to improve the efficiency of the optimal control strategies by including a comorbidity class in the model.

Can we say something about the immunity period provided by a vaccine? Does a better knowledge provides with more elements to establish a better control strategy? 

We need to include a short discussion of the usual way a control is implemented and the main difference with the approach in this work.