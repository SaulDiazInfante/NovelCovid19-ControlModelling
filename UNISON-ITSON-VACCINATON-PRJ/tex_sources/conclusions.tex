
At the date of writing this article, humankind lacks strategies to
eradicate COVID-19.  Although NPIs implemented in most countries prevent
citizens from being infected, these strategies leave them
susceptible\textemdash people can not develop immunity to face futures waves.
Thus, vaccination becomes the primary pharmaceutical measure to recover life's
style before the pandemic. However, this vaccine has to be effective and well
implemented in global
vaccination programs. Thus new challenges as its distribution, stocks,
politics, vaccination efforts, among others, emerge. A fair distribution and
application strategy is imperative to manage the available resources,
especially in developing countries.

%\paragraph{Statement of principal findings}
We established an optimal control problem to design vaccination strategies
where vaccination modulates dynamics susceptibility through an imperfect
vaccine. We aimed to provide vaccination policies that minimize the lost life
years due to disability or premature death by COVID-19, determined by
cumulative deaths and cumulative incidence. Policies' acts in the minimization
of infected people's prevalence and the number of deaths.

%\paragraph{Strengths and weaknesses of the study}
Our simulations suggest a better response with optimal vaccination policies
than policies with a constant vaccination rate. For example, the optimal policy
schedule in scenario \textbf{SCN-1} increases the number of doses in the
scheme's initial stage. This vaccination scheme improves the mitigation of the
symptomatic prevalence, the incidence of deaths, and in consequence, the years
of life lost quantified in DALYs.
%

Emerging press releases reported that Pfizer's, Russian Sputnik V,
and Moderna's coronavirus vaccines reach efficacy over \SI{90}{\percent}
\cite{cnn_health_2020,reuters2020, cnn_health_2020b}. However,
this information remains under development. Thus vaccine efficacy scenarios of
\SI{50}{\percent}, \SI{70}{\percent}, and \SI{90}{\percent} \textbf{(SCN-2)}
illustrate the effect on optimal vaccination policies' schedule by pointing
when to intensify the number of doses. According to the time horizon of one
year and coverage of 50\%, our numerical experiments suggest that
\SI{90}{\percent} vaccine efficacy reduces around three times the number of
deaths regarding the dynamics without vaccination. Likewise, these vaccines
reach a gain of eighteen times in the years of life lost compared to the
without vaccination scenario.


Our numerical experiments also illustrate vaccine-induced immunity's
relation between the reproductive vaccination number $R_V$ and vaccination
policies \textbf{(SCN-3)}. Considering an outbreak with a reproductive number
$R_0$ of \num{1.794 93}, vaccine-induced immunity of \num{365} or
\SI{730}{days} implies a reduction of $R_0$\textemdash dropping its value
respectively to  \num{1.139130} and \num{.86756}.
Likewise, optimal policies linked to vaccine-induced immunities enhance
symptomatic prevalence mitigation and the number of saved lives. Moreover,
according to the initial number of deaths, the scenario without vaccination
accumulates \num{503} deads compared to \num{211} and \num{206} deads of the
underlying dynamics with vaccine-induced immunities.

%\paragraph{Strengths and weaknesses about other studies, discussing
%significant
%differences in results}
%
Barbosa et al. recently report in \cite{Barbosa2020} a simpler model about
modeling of COVID-19 vaccination with a similar approach. Although they
establish a less detailed model \textemdash they do not distinguish between
symptomatic and asymptomatic infected individuals\textemdash its control
problem return policies according to multi-objective policies. Their optimal
policy is pragmatic but, in our opinion, not necessarily practical for large
populations. Further, our model extends the result of \cite{Barbosa2020} by a
more detailed vaccine profile. Thus we can evaluate how vaccine efficacy,
vaccine-induced, and natural immunity parameters impact the mitigation of an
optimal vaccination schedule.

Perkins and España report in \cite{Perkins2020} a vaccination model with
optimal control, but they approach to optimize the NPIs. The methodology
presented here is similar, but aim very different. However, we want to stress
the relevance of also including NPIs effects.

%\paragraph{Meaning of the study: possible explanations and implications for
%clinicians and policymakers}
Since any vaccine's efficacy will be subject to uncertainty and immunization
regarding COVID-19 remains under development, policymakers need better modeling
tools to design fair vaccination programs. We faced this problem by simulation.

%\paragraph{Unanswered questions and future research}
According to DALYs definition, segregation as age, comorbidities, and other
risk groups is imperative to design more realistic vaccination policies.
Moreover, it is well-known that various vaccines platforms and strategies are
developing in parallel, and the most recent advance is with vaccines that
require two doses. From \cite{Perkins2020}, we can deduce that NPIs, together
with
vaccination, would constitute a better description of COVID-19 control. We will
direct our attention to extend this work according to the segregation and
optimization of NPIs-vaccination controls.


