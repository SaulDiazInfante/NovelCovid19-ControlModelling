    In late December 2019, a new virus's appearance is reported in Wuhan City,
Hubei Province, China. Called SARS-CoV2, it is the virus that causes the 2019
coronavirus disease (COVID-19) and that, very quickly since its appearance, has
spread throughout much of the world, causing severe problems to health systems
of all the countries in which it is present \cite{Who12020}.

    Due to the absence of successful treatment and vaccines, several
non-pharmaceutical interventions (NPIs) have been implemented in all the
countries where the disease is present, with quarantine, isolation, and social
distancing being the main ones \cite{Wilder2020,Liu2020_2}. Despite the measures
that different governments have taken to mitigate the epidemic, it has not been
controlled in most places, which may be due to the relaxation of mitigation
measures. At the date of writing this work, the upturn or regrowth in the number
of cases in some countries around the world has been observed. In some places,
this behavior is referred to as "second wave". On the other hand, since the new
coronavirus appearance, the international scientific community has been working
to understand the virus nature. They mainly focus on the spreading mechanisms
between individuals, and developing vaccines and treatments to reduce the number
of infections and fatality cases.

    Development of COVID-19 vaccines is the major challenge of these days. Some
research efforts in this direction can be found in \cite{Belete2020,Kaur2020}.
In Mexico, some of the considered vaccines to be applied to the population are
Adenovirus Type 5 Vector (Ad5-nCoV) by Cansino Biologics, AZD1222 by
AstraZeneca and BNT162b2 by Pfizer and BioNTech. At the current date, these
and other vaccines are on the testing phase (phase 3) and it is believed that
their distribution will begin at the end of March 2021. However, there are
still somequestions about the efficacy of the vaccines. The U. S. Food and Drugs
Administration (FDA) requires firm evidence that a vaccine protects at least
half of those inoculated \cite{Shah2020}. At the date, there are no conclusive
results about the vaccines' efficacy, nor the immunity time induced by vaccines.

    To get a clearer understanding of different vaccination strategies and their
consequences on the number of infected individuals, Kermack--McKendrick type
models have taken a leading role. This kind of models have been used to
understand vaccination dynamics on other diseases \cite{Alexander2004}. It is
important to stress that nowadays Kermack--McKendrick type mathematical models
has helped to describe COVID-19 epidemics properties around the globe. These
models have been used to estimate the basic reproductive number associated with
the disease and also different parameters involved in its spread
\cite{Liu2020,Sarkar2020}. Another use of this kind of model has been addressed
to propose and evaluate the effect of various control measures classified as
NPIs
\cite{Acuna2020,Santana2020,Ngonghala2020,Liu2020,Shaikh2020,DeVisscher2020}.

    On the other hand, vaccines' development for COVID-19 has led to ask which
are the best vaccination strategies to reduce the disease levels in a
population. In order to find a good vaccination strategy, a widely used tool is
the optimalcontrol theory. This mathematical method together with
Kermack--McKendrick models are useful to propose scenarios in which the
vaccine's application minimizes the damage caused by the disease and its
application cost. Some results about it have been applied to control other
diseases for humans and animals
\cite{Asano2008,Rodrigues2014,Malik2016,Jaberi2014}. Optimal control theory has
also been used in COVID-19 studies. Most efforts have been invested in finding
optimal strategies to evaluate the impact of non-pharmaceutical interventions
\cite{Madubueze2020,Perkins2020,Ullah2020}. Optimal control strategies have
also being considered in vaccination strategies \cite{Barbosa2020}. In this
work, the authors took their optimization based on an extension of the SIR
mathematical model for COVID-19, which included a vaccinated class. Within
their objectives was to minimize only the number of infected individuals
together with the prescribed vaccine concentration during treatment.

    In this work, we present a mathematical model to describe the transmission
and some vaccination dynamics of COVID-19. The aim is to implement optimal
control theory to obtain vaccination strategies in a homogeneous population.
Such strategies are aligned to the policies of the WHO strategic advisory group
of experts (SAGE) on COVID-19 vaccination \cite{sage2020}. In this sense we
look to minimize the disability-adjusted life year (DALY) \cite{WhoDALY}. This
quantity is used by WHO to quantify the burden of disease from mortality and
morbidity, which is given by the sum of the years of life lost (YLL) and years
lost due to disability (YLD). Motivated by this, we arrive to a different
optimal control problem as the one studied in \cite{Barbosa2020}. In
particular, we focus on vaccination efficacy, natural immunity and a given
coverage for a particular horizon time, to find an optimal daily vaccination
rates strategy. By following this idea, we can (i) find strategies such that
hospitals are not surpassed in their capacity, (ii) value a strategy by
obtaining estimates of the number of infected and deceased individuals and
(iii) make comparisons between a constant vaccination policy versus the optimal
vaccination policy.

    Our manuscript is divided into the following sections.
In \Cref{Sec:MathematicalModelFormulation}, we present our mathematical model,
which includes preventive vaccine dynamics. \Cref{Sec:Rv_Analysis} includes an
analysis of the vaccination reproductive number. In
\Cref{Sec:OptimalVaccinePolicies}, we define our optimal control problem when
modulating the vaccination rate by a time-dependent control signal $u_V(t)$.
\Cref{Sec:NumericalExperiment} presents our numerical results regarding optimal
vaccination policies. We end with a conclusions and discussion section.