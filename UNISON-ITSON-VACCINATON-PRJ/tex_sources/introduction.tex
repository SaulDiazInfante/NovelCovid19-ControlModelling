In late December 2019, a new virus's appearance is reported in Wuhan City, Hubei Province, China. Called SARS-CoV2, it is the virus that causes the 2019 coronavirus disease (COVID-19) and that, very quickly since its appearance, has spread throughout much of the world, causing severe problems for health systems of all the countries in which it is present \cite{Who12020}. On March 11, 2020, the World Health Organization declared the epidemic by COVID-19 as a pandemic.  Around 118,000 cases at that time, distributed in 114 countries and with around 4,291 deaths \cite{Who512020}. In Latin America, the first
detected case of COVID-19 occurred in Brazil on February 26, and in Mexico, the first case was reported on February 27, quickly spreading throughout the country \cite{Ajrm2020,Acuna2020}.

Various control measures have been implemented in all the countries where the disease is present, with quarantine, isolation, and social distancing being the main ones. Despite the measures that different governments have taken to mitigate the epidemic, it has not been controlled. The number of cases and deaths from the disease continues to increase in many countries of the world. On the other hand, since the new coronavirus's appearance, the international scientific community has been working to understand the nature of the virus. They mainly focus on the spreading mechanisms between individuals, developing vaccines and treatments to reduce the number of infections and fatality cases. To get a clearer understanding of different vaccination strategies and their consequences on the number of infected individuals, mathematical models have taken a leading role. The use of various mathematical tools such as SIR and SEIR models has helped to describe epidemics properties around the globe.

Initially, these models have been used to estimate the basic reproductive number associated with the disease and estimate different parameters involved in its spread, such as contagion rates, incubation periods, and recovery rates. They are also being used to propose and evaluate the effect of various control measures, such as quarantine \cite{Acuna2020,Wang2020,Marimuthu2020,Liu2020,Ghosh2020,Calvetti2020,Shaikh2020}.

On the other hand, a widely used tool to propose optimal scenarios for applying vaccines or treatments is the optimal control theory. With this tool, it is possible to find scenarios in which the application of the vaccine minimizes the damage caused by the disease and the cost of its application. Some results about it have been applied to control other diseases for humans and animals \cite{Asano2008,Rodrigues2014,Tchuenche2011,Malik2016,Jaberi2014}.

In this work, we present a mathematical model to describe the propagation disease and vaccination dynamics of COVID-19. An analysis of the basic reproductive number is shown, providing conditions, in terms of the vaccination parameters, that allow both reducing the value of $\mathcal{R}_{0}$ and lowering its value below one. Finally, optimal scenarios for applying a preventive vaccine are also presented through optimal control theory. It is important to stress that we used the Mexican Ministry of Health data to estimate the proposed model's parameters before the vaccine application. In this way, we have the possibility of analyzing the effect of the vaccine, maintaining a base of values for the model parameters, and thus being able to manipulate only the parameters corresponding to vaccination. 
