In late December 2019, a new virus's appearance is reported in Wuhan City,
Hubei Province, China. Called SARS-CoV2, it is the virus that causes the 2019
corona virus disease (COVID-19) and that, very quickly since its appearance,
has spread throughout much of the world, causing severe problems to health
systems of all the countries in which it is present \cite{Who12020}. On March
11, 2020, the World Health Organization declared the epidemic by COVID-19 as a
pandemic when there were around 118,000 cases distributed in 114 countries, and
approximately 4,291 deaths \cite{Who512020}. In Latin America, the first
detected case of COVID-19 occurred in Brazil on February 26, and in Mexico, the
first case was reported on February 27, quickly spreading throughout the
country \cite{Ajrm2020,Acuna2020}.

Due to the absence of successful treatment and vaccines,
several non-pharmaceutical interventions (NPIs) have been implemented in all the
countries where the disease is present, with quarantine, isolation, and social
distancing being the main ones \cite{Wilder2020,Guner2020,Liu2020_2}. Despite
the measures that different governments have taken to mitigate the epidemic, it
has not been controlled in most places, which may be due to the relaxation of
mitigation measures. At the date of writing this work, the upturn or regrowth
in the number of cases in some countries around the world has been observed. In
some places, this behavior is referred to as "second wave". On the other hand,
since the new coronavirus appearance, the international scientific community
has been working to understand the virus nature. They mainly focus on the
spreading mechanisms between individuals, and developing vaccines and
treatments to reduce the number of infections and fatality cases. To get a
clearer understanding of different vaccination strategies and their
consequences on the number of infected individuals, mathematical models have
taken a leading role.

    The use of various mathematical tools such as SIR and SEIR models has
helped to  describe epidemics properties around the globe. These models have
been used to estimate the basic reproductive number associated with the disease
and also different parameters involved in its spread
\cite{Wang2020,Liu2020,Sarkar2020,Hong2020}. Another use of this kind of model
has been addressed to propose and evaluate the effect of various control
measures classified as NPIs
\cite{Acuna2020,Marimuthu2020,Liu2020,Shaikh2020,DeVisscher2020}.

    Currently, vaccine development for COVID-19 is at an advanced stage. It is
believed that their distribution will begin in early 2021. However, we
consider that, in addition to the vaccine's existence, it is necessary to
have a good vaccination strategy. A widely used tool to address this
question is the optimal control theory. This mathematical tool is useful to
propose scenarios in which the vaccine's application minimizes the damage
caused by the disease and its application cost. Some results about it have
been applied to control other diseases for humans and animals
\cite{Asano2008,Rodrigues2014,Tchuenche2011,Malik2016,Jaberi2014}. Optimal
control theory has also been used in COVID-19 studies. Most efforts have
been invested in finding optimal strategies to evaluate the impact of
non-pharmaceutical interventions
\cite{Madubueze2020,Perkins2020,Ullah2020}. Optimal control strategies have
also being considered in vaccination \cite{Barbosa2020}. In this work, the
authors took their optimization based on a basic compartmental model for
COVID-19. In their model, deaths due to disease are not considered and
vaccination is applied only to susceptible individuals.  \unsure{NO SE VE
    LA DIFERENCIA RESPECTO A NUESTRO TRABAJO, EL QUE NO CONSIDEREN MUERTE NO LO
    HACE TAN DIFERENTE. RECUERDO QUE SAÚL MENCIONÓ ALGO SOBRE QUE LA FORMA DE
    HACER CONTROL ES DISTINTA. O DE PLANO NO HAY? JAJA....Ohh pues...dejame
    ser! estoy en eso!}

In this work, we present a mathematical model to describe the transmission and some vaccination dynamics of COVID-19. Mainly, we focus on using optimal control theory to obtain vaccination strategies in a homogeneous population. Our main objective is to minimize the disability-adjusted life year (DALY) \cite{WhoDALY}. This quantity is used by the World Health Organization (WHO) to quantify the burden of disease from mortality and morbidity, which is given by the sum of the years of life lost (YLL) and years lost due to disability (YLD). This work's objective is framed within the context of the WHO strategic advisory group of experts (SAGE) on immunization working group on COVID-19 vaccines \cite{sage2020}.

Development of COVID-19 vaccines is a major challenge these days. Some research efforts in this direction can be found in \cite{Belete2020,Kaur2020}. In Mexico, some of the considered vaccines to be applied to the population are Adenovirus Type 5 Vector (Ad5-nCoV) by Cansino Biologics, AZD1222 by AstraZeneca and BNT162b2 by Pfizer and BioNTech. At the current date, these and other vaccines are on the testing phase (phase 3) and there are still some questions about the efficacy of the vaccines. In the United States, an appropriate vaccine to be applied needs to show firm evidence that it protects at least half of those inoculated \cite{Shah2020}. At this stage, there are no conclusive results about the efficacy of the vaccines. Also, current work is the immunity time the vaccines will provide to people.

Our work is divided into the following sections. In Section 2, we present our mathematical model, which includes preventive vaccine dynamics. Section 3 includes an analysis of the vaccination reproductive number. Section 4 presents our numerical results regarding optimal vaccination policies. It is important to stress that, with the objective of study COVID-19 dynamics in a specific city and have a set of baseline parameters values, we used the Mexico City plus Mexico state COVID-19 data to estimate some proposed model parameters. We end this work with a conclusions and discussions section.