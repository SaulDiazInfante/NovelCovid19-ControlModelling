\noindent In this section, we formulate our baseline mathematical model, which additionally to the transmission dynamics, includes vaccination. In order to build our model, we follow the classical Kermack-McKendrick approach. Our model splits the population in seven different classes: Susceptible $(S)$, exposed $(E)$, symptomatic infected $(I_S)$, asymptomatic infected $(I_A)$, recovered $(R)$, death $(D)$ and vaccinated $(V)$ individuals. It is important to mention that $I_{S}$ represents the proportion of symptomatic individuals who will later report to some health medical center. Additionally to the classical process an individual might follows during infection, in our model, we assume that reinfection is possible after a period of time. We model the vaccination process considering some assumptions: i) Vaccination is applied to all the alive individuals except those in the symptomatic class. In this situation, vaccines are distributed over individuals on the $S$, $E$, $I_A$, and $R$ classes. ii) The vaccine has preventive nature, that is, only reflected in the susceptible individuals $(S)$. iii) People will only get one vaccine during the campaign. iv) Vaccines do not necessarily have a hundred percent of effectivity, which implies that some vaccinated people can get the disease. We denote the effectivity rate by $\epsilon$. Based on these assumptions our model becomes

\begin{equation}\label{model1}
  \begin{aligned}
	S'(t)&=\mu \bar{N}-\frac{\beta_S I_S+\beta_AI_A}{\bar{N}}S-(\mu+\lambda_V)S +\delta_V V+ \delta_R R\\
	E'(t)&= \frac{\beta_S I_S+\beta_AI_A}{\bar{N}}S+(1-\epsilon) \frac{\beta_S I_S+\beta_AI_A}{\bar{N}}V-(\mu+\delta_E) E \\
	I'_S(t)&= p \delta_E E-(\mu+\alpha_S) I_S\\
	I'_A(t)&= (1-p) \delta_E E-(\mu+\alpha_A) I_A \\
	R'(t)&= (1-\theta) \alpha_S I_S+\alpha_A I_A-(\mu+\delta_R) R \\
	D'(t)&= \theta \alpha_S I_S \\
	V'(t)&= \lambda_V S-(1-\epsilon)  \frac{\beta_S I_S+\beta_AI_A}{\bar{N}}V-(\mu+\delta_V) V
      \end{aligned}
\end{equation}
where $\bar{N}(t)=S(t)+E(t)+I_S(t)+I_A(t)+R(t)+V(t)$ and $N=\bar{N}+D$. Parameters description of system~\ref{model1} is given in Table \ref{table1}.
\begin{table}[h!]
    \begin{tabular}{>{\centering}p{0.2\textwidth}p{0.6\textwidth}}
			\toprule
			Parameter & Description
      \\
      \midrule
			$\mu$ &  Death rate
			\\
            $\beta_S$ & Infection rate between susceptible and symptomatic infected
			\\
            $\beta_A$ & Infection rate between susceptible and asymptomatic infected
			\\
            $\lambda_V$ & Vaccination rate
			\\
            $\delta_{V}^{-1}$ & Immunity average time by vaccination
			\\
            $\epsilon$ &  Effectivity rate of the vaccination
			\\
            $\delta_{E}^{-1}$ & Average time of the incubation period \\
			$p$ & Proportion of symptomatic individuals  \\			
            $\alpha_{S}^{-1}$ &  Average output time of symptomatic individuals due to death or recovery  \\
            $\theta$ & Proportion of symptomatic individuals who die due to the disease \\ 
			$\alpha_{A}^{-1}$ & Recovery average time of asymptomatic individuals  \\ 					 
            $\delta_{R}^{-1}$ &  Immunity average time by disease \\
      \bottomrule
		\end{tabular}
  \caption{Parameters definition of system~\ref{model1}.}\label{table1}
\end{table}
%\pagebreak
