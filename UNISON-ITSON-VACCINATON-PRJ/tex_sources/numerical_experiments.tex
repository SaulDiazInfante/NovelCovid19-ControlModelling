According to the WHO Strategic Advisory Group of Experts (SAGE) on 
Immunization Working Group on COVID-19 Vaccines modeling questions 
presented in \cite{sage2020}, our model is capable to explore the following scenarios:
    \subsection*{Vaccine profile}
    Vaccination policies according to different profiles, namely:
    \begin{itemize}
      \item
          efficiency
          $\epsilon = \{\num{0.1}, \num{0.2}, \cdots  \num{0.9} \}$ .
      \item
        vaccine induced immunity
        $$
          \delta_v ^{-1}=
            \{\SI{0.5}{year},
              \SI{1.0}{year}, \cdots, \SI{}{lifelong}
            \}.
        $$
    \end{itemize}


  \subsection*{Coverage}
    We obtained optimal vaccination policies with coverage profiles at
    $$
      x_{coverage} =
        \{
          20\%, 50\%, 80\%
        \}.
    $$
  \subsection*{Time horizon}
  Plausible scenarios at a different analytic time horizon
  $$
    T= \{ \SI{1}{year}, \SI{2}{year}, \SI{10}{year} \}.
  $$
  
 To fix ideas,  we display in \Cref{fig:bocop_scene} the counterfactual 
scenario regarding no intervention, constant vaccination policy (CP), and 
optimal vaccination policy (OP). Dashed lines denote the prevalence of each 
class according to dynamics with no intervention. Solid lines represent the 
prevalence according to controlled dynamics with the optimal or constant 
vaccination policies. Thus, shaded areas respectively denote the gain in 
mitigation(orange), health resources (red), saved lives (green), coverage 
(blue), and cost(olive). Here, opaque colors correspond to a constant 
vaccination policy while translucent colors are related to the gain 
according to the optimal policy. For example, in the panel titled "Death,"  
the opaque solid line represents the number of accumulated deaths with a 
constant vaccination policy. The green opaque shaded area is the number of 
saved lives regarding to CP. Since the translucent green shade area overlaps the opaque 
green, we say that optimal vaccination improves the gain of a constant 
policy. Further, since the cost (see olive color) of the CP is above OP, we 
also say that optimal vaccination is cheaper.  
\begin{figure}[h!]
  \includegraphics[width=\textwidth]{fig1.pdf}
  \caption{Counterfactual scenario regarding no intervention, constant vaccination policy (CP), and 
optimal vaccination policy (OP), with vaccination efficiency $\epsilon = 0.7$, induced vaccination immunity 
$\delta_V= 300$ days and vaccination base rate $\lambda_V = 50 k/day$.}
  \label{fig:bocop_scene}
\end{figure}


In \Cref{fig:bocop_scene}, we show a scenario where $\mathcal{R}_0>1$. Despite $\mathcal{R}_v$ remains below but close to $\mathcal{R}_0$.
However, vaccination policies improve the estimated damage.

Vaccination reproductive number $\mathcal{R}_v$, suggests constant policies 
according to the mitigation factor
$$
    \left(
        1 - 
        \frac{\epsilon \lambda_V}{\mu+\delta_V+\lambda_V}
    \right).
$$    
Then disease mitigation is strongly related to vaccine efficiency
$\epsilon$ and vaccination rate $\lambda_v$. Further, given a dynamic with 
not vaccine intervention and $\mathcal{R}_0>1$, $\mathcal{R}_v$ suggests a minimal vaccination rate to drive this dynamic to the disease-free state. 
In particular, this vaccine efficiency would govern the viability of the 
constant policy. 



