\begin{abstract}
    %BACKGROUND
        At the date, Europe and part of North America face the second wave of
    COVID-19, causing more than \num{1300000} deaths worldwide. Humanity lacks
    successful treatments, and a sustainable solution is an effective vaccine.
    Pfizer and the Russian Gamaleya Institute report that its vaccines reach
    more than \SI{90}{\percent} efficacy in a recent press release. If third
    stage trial results favorable, pharmaceutical firms estimate big scale
    production of its vaccine candidates around the first 2021 quarter and the
    World Health organization fix as objective, vaccinate \SI{20}{\percent} of
    the whole population at the final of 2021. However, since COVID-19 is new to
    our knowledge, vaccine efficacy and induced-immunity responses remain poorly
    understood. There are great expectations, but few think the first vaccines
    will be fully protective. Instead, they may reduce the severity of illness,
    reducing hospitalization and death cases.

    %PROBLEM SETUP
        Further, logistic supply, economic and political implications impose a
    set of grand challenges to develop vaccination policies. For this reason,
    health decision-makers require tools to evaluate hypothetical scenarios and
    evaluate admissible responses.

    %FINDINGS
        Our contribution answers questions in this direction. According to the
    WHO Strategic Advisory Group of Experts on Immunization Working Group on
    COVID-19 Vaccines, we formulate an optimal controlled model to describe
    vaccination policies that minimize the burden of COVID-19 quantified by the
    number of disability-adjusted years of life lost.  Additionally, we analyze
    the reproductive vaccination number according to vaccination profiles
    depending on coverage, efficacy, horizon time, and vaccination rate. We
    explore scenarios regarding efficacy, coverage, vaccine-induced immunity,
    and natural immunity via numerical simulation. Our results suggest that
    response regarding vaccine-induced immunity and natural immunity would play
    a dominant role in the vaccination policy design.  Likewise, the vaccine
    efficacy would influence the time of intensifying the number of doses in the
    vaccination policy.
\end{abstract}
