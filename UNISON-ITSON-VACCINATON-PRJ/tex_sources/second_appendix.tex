

\begin{lemma}
    The set  $\Omega=\{(S,E,I_S,I_A,R,D,V)\in \R^{7}_{+}: S+E+I_S+I_A+R+D+V=N\}$ is a 
    positively invariant set for the system (\ref{model1}).
\end{lemma}

\begin{proof}
To prove the lemma 1, let $\Omega=\{(S,E,I_S,I_A,R,D,V)\in \R^{7}_{+}: S+E+I_S+I_A+R+D+V=N\}$. First, note that for this model we have a closed population, which allows the solutions to be bounded superiorly by the total population. 

On the other hand, to show the positivity of the solutions with initial conditions $$(S (0), E (0), I_S (0), I_A (0), R (0), D (0), V (0)) \in \R^{7}_{+},$$ we look at the direction of the vector field on the hypercube faces in the direction of each variable in the system. For example, consider a point on the hypercube face where the variable $S = 0$ and look at the behavior of the vector field in the direction of the same variable $S$, to see if the solutions cross the face of the hypercube where we are taking the initial condition. So, notice that if $ S = 0 $, $ S '(t) > 0 $, so the solution points into the hypercube. Similarly, consider an initial condition of the form $ (S, 0, I_S, I_A, R, D, V) $ and note that $ E '(t)> 0 $ for all $ t> 0 $, which implies that the solutions of the system with initial conditions of the form $  (S, 0, I_S, I_A, R, D, V)  $ point towards the interior of the hypercube. Similarly, positivity can be tested for the rest of the variables. With this information, we have the following result.
\end{proof}


Now, continuing with the analysis of our model, it is easy to prove that the disease-free equilibrium is given by the point at $ X_0 \in \Omega $ of the form
$$
    X_0 =\left(
            \frac{(\mu+\delta_V)\bar{N}}{\mu+\delta_V+\lambda_V}, 
            0, 0, 0, 0,
            \frac{\lambda_V \bar{N}}{\mu+\delta_V+\lambda_V}
    \right).
$$
On the other hand, following the ideas in \cite{Diekmann1990, Van2002}, the next generation matrix for this model, evaluated in the disease equilibrium point, is given by

\begin{align}\label{NGM1}
    \begin{split}
    \textbf{K}=
    \begin{bmatrix}
        \frac{\delta_E}{(\mu + \delta_E)}
        \left(\frac{p\beta_S}{\mu + \alpha_S + \mu_S + \lambda_T} \right) (S^* + \epsilon V^*) 
        & \frac{\beta_S (S^* + \epsilon V^*)}{(\mu + \alpha_S + \mu_S + \lambda_T)\bar{N}} 
        & \frac{\beta_A (S^* + \epsilon V^*)}{(\mu + \alpha_A + \mu_A) \bar{N}}
        \\
        0 & 0 & 0 \\
        0 & 0 & 0  \\
    \end{bmatrix}
\end{split}
\end{align}
where $S^*=\frac{(\mu+\delta_V)\bar{N}}{\mu+\delta_V+\lambda_V}$ 
and $V^*=\frac{\lambda \bar{N}}{\mu+\delta_V+\lambda_V}$. Then, her spectral ratio is

\begin{equation*}
    R_{V}=R_S+R_A
\end{equation*}
with

\begin{eqnarray*}
    R_S &=& \frac{p\beta_S\delta_E(\mu+\delta_V+(1-\epsilon)                 
            \lambda_V)}{(\mu+\delta_E)(\mu+\delta_V+\lambda_V)(\mu+\alpha_S+\mu_S+\lambda_T)}
    \\ 
    R_A &=& \frac{(1-p) \beta_A \delta_E (\mu + \delta_V + (1-\epsilon)
            \lambda_V)}{(\mu+\delta_E)(\mu + \delta_V + \lambda_V)(\mu + \alpha_A + \mu_A)}
\end{eqnarray*}




