
Mathematical models for COVID-19 have shown that the
parameters' values are not necessarily the same in each country. We use
COVID-19 data from Mexico City and Mexico state to follow the epidemic
curve's initial growth in this work. Consequently, we estimate some
parameter values of system~\eqref{model1}. To obtain the baseline
parameter values, we consider two-stages: i) before and ii) after
mitigation measures were implemented. For both stages, we use
model~\eqref{model1} with no vaccination dynamics ($\lambda_V = 0$ and
$V(0) = 0$), and STAN R-package. This package is used for statistical
inference by the Bayesian approach. For the code implementation of our
system, we follow the ideas of \cite{Chatzilena2019}, and it is made
freely available at... For this section, our estimations are focused on
three parameters: $\beta_A$, $\beta_S$ and $p$. Other parameter values
are given in Table~\eqref{table_fixparam}.
\begin{table}[h!]
\begin{center}
    \begin{tabular}{ccc}
    \toprule
        Parameter & Value & References
        \\
        \midrule
        $\delta_{E}^{-1}$ & $5.1\ \text{days}$   &  \cite{Tian2020}
        \\
        $\alpha_{S}^{1}$  & $5.97\ \text{days}$  &  \cite{Acuna2020}
        \\
        $\alpha_{A}^{-1}$ & $10.81\ \text{days}$ & \cite{Acuna2020}
        \\
        $\delta_{R}^{-1}$ & $365\ \text{days}$     &
        \\
        $\mu^{-1}$        & $70\ \text{years}$   &
        \\
        \bottomrule
    \end{tabular}
        \caption{Fixed parameters values of
    system~\eqref{model1}.}\label{table_fixparam}
    \end{center}
\end{table}

For the first stage, the following system is considered:
\begin{equation}\label{model_stage1}
  \begin{aligned}
	S'(t)&=\mu \bar{N}-\frac{\hat{\beta}_S
	I_S+\hat{\beta}_AI_A}{\bar{N}}S-\mu S + \delta_R R\\
	E'(t)&= \frac{\hat{\beta}_S I_S+\hat{\beta}_A
	I_A}{\bar{N}}S-(\mu+\delta_E) E \\
	I'_S(t)&= p \delta_E E-(\mu+\alpha_S) I_S\\
	I'_A(t)&= (1-p) \delta_E E-(\mu+\alpha_A) I_A \\
	R'(t)&= (1-\theta) \alpha_S I_S+\alpha_A I_A-(\mu+\delta_R) R \\
	D'(t)&= \theta \alpha_S I_S
  \end{aligned}
\end{equation}
where $\bar{N}(t)=S(t)+E(t)+I_S(t)+I_A(t)+R(t)$ and $N=\bar{N}+D$. Here,
we consider COVID-19 data from the first day of symptoms onset reported
(February 19) until March 23, 2020. We also assume that $\theta = 0$
because the first reported death was on March 18, and there were three
reported deaths until March 23. The initial values of recovered and dead
people are set to zero. Symptomatic class initial value was fixed in one
individual, while $E(0)$ and $I_{A}(0)$ were estimated. Thus, $S(0) = N -
(E(0) + I_{A}(0) + 1)$, where $N = 26446435$ \cite{conavi2020}. For the
STAN implementation, we employ a negative-binomial model as the
likelihood function with the mean parameter given by incidence solution
per day (FALTA). In addition to the above, we assign prior probability
distributions to each parameter and the exposed and asymptomatic classes'
initial conditions. Thus, we propose that $\hat{\beta}_A$ and
$\hat{\beta}_S$ follow a normal distribution with parameters $\mu = 1$
and $\sigma^2 = 0.13$. Then,  $p$ follows a uniform distribution in $(0,
0.25)$, and $E(0)$ and $I_{A}(0)$ also follow a uniform distribution in
$(2,20)$ and $(2,10)$, respectively. When employing our STAN
implementation, we run 5 chains with 100,500 iterations each, discard the
first 500, and use 10,000 samples to generate estimates of parameters
$\hat{\beta}_A$, $\hat{\beta}_S$ and $p$. Table~\eqref{table_icparam}
shows the confidence interval for each parameter and median posterior
estimated.
\begin{table}[h!]
\begin{center}
	\begin{tabular}{ccc}
		\toprule
	    Parameter & 95\% Confidence Interval & Quantile 50
			\\
			\midrule
            $\hat{\beta}_S$ & $[0.672, 1.1886]$   &  $0.9322$ \\
            $\hat{\beta}_A$ & $[0.501, 0.7851]$  &  $0.6435$ \\
            $p$       & $[0.061, 0.2206]$ &  $0.1227$ \\
            $R_0$ & $[4.159, 5.1991]$ &  $4.6082$ \\
			\bottomrule
	\end{tabular}
  \caption{Confidence interval and median posterior estimated for some
  parameters of system~\eqref{model_stage1}and basic reproductive number
  $(R_0)$.}\label{table_icparam}
\end{center}
\end{table}

\noindent For the second stage, we took a complete month starting the day
when mitigation measures were implemented, that is, from March 23 to
April 23, 2020. Now, we consider parameter $\xi$ to model the
implementation of non-pharmaceutical measures. Thus,
system~\eqref{model_stage1} becomes:
\begin{equation}\label{model_stage2}
  \begin{aligned}
	S'(t)&=\mu \bar{N}-\frac{\xi\hat{\beta}_S
	I_S+\xi\hat{\beta}_AI_A}{\bar{N}}S-\mu S + \delta_R R\\
	E'(t)&= \frac{\xi\hat{\beta}_S
	I_S+\xi\hat{\beta}_AI_A}{\bar{N}}S-(\mu+\delta_E) E \\
	I'_S(t)&= p \delta_E E-(\mu+\alpha_S) I_S\\
	I'_A(t)&= (1-p) \delta_E E-(\mu+\alpha_A) I_A \\
	R'(t)&= (1-\theta) \alpha_S I_S+\alpha_A I_A-(\mu+\delta_R) R \\
	D'(t)&= \theta \alpha_S I_S
  \end{aligned}
\end{equation}
where $\bar{N}(t)=S(t)+E(t)+I_S(t)+I_A(t)+R(t)$ and $N=\bar{N}+D$. At
this stage, we consider that $\theta = 0.11$. Here, our objective is to
estimate the value of parameter $\xi$. To do this, we use the median
posterior of all the estimated parameters from the first stage (see
Table~\eqref{table_icparam}). Other parameter values are given in
Table~\eqref{table_fixparam}. Almost all initial conditions were obtained
when solving system~\eqref{model_stage1} with the 10,000 samples
(obtained in first stage), after which each solution at the final time
(March 23) is saved. We use the median of the saved values. Thus, for
system~\eqref{model_stage2}, $E(0) = 6587.585$, $I_S(0) = 553.7035$,
$I_A(0) = 3149.924$, and $R(0) = 3001.547$. For the initial value of
variable $D$, we consider reported COVID-19 data, then $D(0) = 3$.
Therefore $S(0) = N - (E(0) + I_S(0) + I_A(0) + R(0) + D(0))$, with $N =
26446435$. Similar to the first stage, we consider a negative-binomial
model as the likelihood function with the mean parameter given by
incidence solution per day (FALTA), while that we postulate a uniform
distribution in $(0.25,0.75)$ as a prior probability distribution for the
parameter $\xi$.
\label{App:Parameter_Est}
For the second stage, we run 5 chains with 100,500 iterations each,
discard the first 500, and use 10,000 samples to generate estimates of
parameters $\xi$. Table~\eqref{table_icparam2} shows the confidence
interval and median posterior estimated for parameter $\xi$.
\begin{table}[h!]
\begin{center}
	\begin{tabular}{ccc}
		\toprule
	    Parameter & 95\% Confidence Interval & Quantile 50
			\\
			\midrule
            $\xi$     & $[0.3696, 0.4099]$  & $0.3889$  \\
            $R_0$ & $[1.702, 1.887]$ &  $1.791$ \\
			\bottomrule
	\end{tabular}
  \caption{Confidence interval and median posterior estimated for
  parameter $\xi$ of system~\eqref{model_stage2} and basic reproductive
  number $(R_0)$.}\label{table_icparam2}
\end{center}
\end{table}

Finally, it is important to mention that our results were implemented
considering that the effective transmission contact rates
$(\beta_{\bullet})$ were equal to $\xi\hat{\beta}_{\bullet}$. This last
means that our scenarios consider the first reduction in the effective
transmission contact rates by NIPs. Using values in
Tables~\ref{table_icparam} and~\ref{table_icparam2}, we build confidence
intervals for $\xi\hat{\beta}_{\bullet}$. These results are shown in
Table~\eqref{table_icparam3}.
\begin{table}[h!]
\begin{center}
	\begin{tabular}{cc}
		\toprule
	    Parameter & 95\% Confidence Interval
			\\
			\midrule
            $\beta_S = \xi\hat{\beta}_S$ & $[0.2483712, 0.48720714]$ \\
            $\beta_A = \xi\hat{\beta}_A$ & $[0.1851696, 0.32181249]$ \\
			\bottomrule
	\end{tabular}
  \caption{Confidence interval for parameters
  $\beta_{\bullet}$.}\label{table_icparam3}
\end{center}
\end{table}

%\section{Parameter setting}

%Parameter setting is done in two stages; in both stages,  we started off
%the model (\eqref{model1}), which is simplified by adapting it to the
%stage to be modeled. With the parameters obtained through the
%adjustment,
%the value of the basic reproductive number ($ R_0 $) will be obtained,
%defined as the average number of secondary infections generated by an
%infectious individual, during their entire period of infectivity, when
%in
%contact with a population of fully susceptible individuals.

%For the second stage analyzed, corresponding to the quarantine period in
%Mexico and which includes from March 23 to May 31, 2020, we used the
%model (\eqref{model2}) considering one more class, which corresponds to
%the population of people dying due to disease. Once normalized with
%respect to the population of living individuals, the model is given by:

%\begin{eqnarray}\label{model2}
%S'(t)&=&\mu \bar{N} -\frac{\beta_S I_S+\beta_AI_A}{\bar{N}}S-\mu
%S+\delta_R R %\nonumber \\
%E'(t)&=&\frac{\beta_S I_S+\beta_AI_A}{\bar{N}}S-(\mu+\delta_E) E
%\nonumber\\
%I'_S(t)&=&p \delta_E E-(\mu+\alpha_S) I_S\\
%I'_A(t)&=&(1-p) \delta_E E-(\mu+\alpha_A) I_A \nonumber\\
%R'(t)&=& \theta \alpha_S I_S+\alpha_A I_A-(\mu+\delta_R) R  \nonumber\\
%D'(t)&=&  (1-\theta) \alpha_S  I_S \nonumber
%\end{eqnarray}
%where $\bar{\epsilon}$ represents the reduction factor in infection
%rates due to the quarantine effect.

%For the model, the parameters $ \mu = 0.653 \times 10^{- 6} $ and $
%\delta_E = 0.196078431 $ were set, which correspond to the reciprocals
%of
%the half-life, which is 75 years, and the estimated incubation period in
%5.1 days \cite{peru2}. Now, using the next generation matrix method, the
%basic reproductive number for model (\eqref{model2}) is given by:

%\begin{equation}\label{R02}
%R_{01}=\frac{p\delta_E\beta_S}{(\mu+\alpha_S)(\mu+\delta_E)}+\frac{(1-p)\delta_E\beta_A}{(\mu+\alpha_A)(\mu+\delta_E)},
%\end{equation}



%