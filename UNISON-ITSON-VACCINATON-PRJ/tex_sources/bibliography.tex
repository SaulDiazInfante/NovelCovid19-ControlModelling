\begin{thebibliography}{222}
  \bibitem{Backer} JBacker Jantien A , Klinkenberg Don , Wallinga Jacco .
  Incubation period of 2019 novel coronavirus (2019-nCoV) infections among
  travellers from Wuhan, China, 2028 January 2020. Euro Surveill.
  2020;25(5):pii=2000062.
  https://doi.org/10.2807/1560-7917.ES.2020.25.5.2000062.

  \bibitem{Tang} Biao Tang, Xia Wang, Qian Li, Nicola Luigi Bragazzi, Sanyi
  Tang, Yanni Xiao, Jianhong Wu Estimation of the transmission risk of
  2019-nCov and its implication for public health interventions. Preprint

  \bibitem{Diekman} Diekmann, O., Heesterbeek, J. A. P., \& Metz, J. A.
  (1990). On the definition and the computation of the basic reproduction
  ratio R 0 in models for infectious diseases in heterogeneous populations.
  Journal of mathematical biology, 28(4), 365-382.

  \bibitem{Driessche} Van den Driessche, P., \& Watmough, J. (2002).
  Reproduction numbers and sub-threshold endemic equilibria for compartmental
  models of disease transmission. Mathematical biosciences, 180(1-2), 29-48.

  \bibitem{Slav} HERMANOWICZ, Slav W. Forecasting the Wuhan coronavirus
  (2019-nCoV) epidemics using a simple (simplistic) model. medRxiv, 2020.

  \bibitem{Zhao} Zhao, S., Ran, J., Musa, S. S., Yang, G., Lou, Y., Gao, D., \&
  He, D. (2020). Preliminary estimation of the basic reproduction number of
  novel coronavirus (2019-nCoV) in China, from 2019 to 2020: a data-driven
  analysis in the early phase of the outbreak. bioRxiv 2020; published online
  Jan 24. DOI, 10(2020.01), 23-916395.

  \bibitem{Chen} Chen, Y., Cheng, J., Jiang, Y., \& Liu, K. (2020). A time
  delay dynamical model for outbreak of 2019-nCoV and the parameter
  identification. arXiv preprint arXiv:2002.00418.

  \bibitem{Riou} Riou, J., \& Althaus, C. L. (2020). Pattern of early
  human-to-human transmission of Wuhan 2019 novel coronavirus (2019-nCoV),
  December 2019 to January 2020. Eurosurveillance, 25(4).

  \bibitem{Lancetfeb4} Wang, F. S., \& Zhang, C. (2020). What to do next to
  control the 2019-nCoV epidemic?. The Lancet, 395(10222), 391-393.

  \bibitem{Quilty} Quilty, B. J., \& Clifford, S. (2020). Effectiveness of
  airport screening at detecting travellers infected with novel coronavirus
  (2019-nCoV). Eurosurveillance, 25(5).

  \bibitem{Who} Imai, N., Cori, A., Dorigatti, I., Baguelin, M., Donnelly, C.
  A., Riley, S., \& Ferguson, N. M. (2020). Report 3: transmissibility of
  2019-nCov. Reference Source.

  \bibitem{Kucharski} Funk, S., \& Eggo, R. M. Early dynamics of transmission
  and control of 2019-nCoV: a mathematical modelling study.

  \bibitem{Two} Wu, J. T., Leung, K., \& Leung, G. M. (2020). Nowcasting and
  forecasting the potential domestic and international spread of the 2019-nCoV
  outbreak originating in Wuhan, China: a modelling study. The Lancet,
  395(10225), 689-697.

  \bibitem{Boldog} Boldog, P., Tekeli, T., Vizi, Z., Dones, A., Bartha, F. A., \&
  Rost, G. (2020). Risk assessment of novel coronavirus COVID-19 outbreaks
  outside China. Journal of Clinical Medicine, 9(2), 571.

  \bibitem{Read} Read, J. M., Bridgen, J. R., Cummings, D. A., Ho, A., \&
  Jewell, C. P. (2020). Novel coronavirus 2019-nCoV: early estimation of
  epidemiological parameters and epidemic predictions. medRxiv.

  \bibitem{ghosh} Ghosh, U., Kamrujjaman, M., \& Ghosh, J. K. (2020). Dynamics
  of SEAIQR Model with Saturated Type Treatment: A Case Study of Spain
  COVID-19.

  \bibitem{amjad} Shaikh, A. S., Shaikh, I. N., \& Nisar, K. S. (2020). A
  Mathematical model of COVID-19 using fractional derivative: Outbreak in
  India with dynamics of transmission and control.

  \bibitem{liu} Liu, M., Ning, J., Du, Y., Cao, J., Zhang, D., Wang, J., \&
  Chen, M. (2020). Modelling the evolution trajectory of COVID-19 in Wuhan,
  China: Experience and suggestions. Public Health.

  \bibitem{peru1} Acuna-Zegarra, M. A., Comas-Garcia, A., Hernandez-Vargas,
  E., Santana-Cibrian, M., \& Velasco-Hernandez, J. X. (2020). The SARS-CoV-2
  epidemic outbreak: a review of plausible scenarios of containment and
  mitigation for Mexico. medRxiv.

  \bibitem{peru2} Acuna-Zegarra, M. A., Santana-Cibrian, M., \&
  Velasco-Hernandez, J. X. (2020). Modeling behavioral change and COVID-19
  containment in Mexico: A trade-off between lockdown and compliance.
  Mathematical Biosciences, 108370.

  Rst, G. (2020). Risk assessment of novel coronavirus COVID-19 outbreaks
  outside China. Journal of Clinical Medicine, 9(2), 571.

  \bibitem{Read} Read, J. M., Bridgen, J. R., Cummings, D. A., Ho, A., \&
  Jewell, C. P. (2020). Novel coronavirus 2019-nCoV: early estimation of
  epidemiological parameters and epidemic predictions. medRxiv.

  \bibitem{ghosh} Ghosh, U., Kamrujjaman, M., \& Ghosh, J. K. (2020). Dynamics
  of SEAIQR Model with Saturated Type Treatment: A Case Study of Spain
  COVID-19.

  \bibitem{amjad} Shaikh, A. S., Shaikh, I. N., \& Nisar, K. S. (2020). A
  Mathematical model of COVID-19 using fractional derivative: Outbreak in
  India with dynamics of transmission and control.

  \bibitem{liu} Liu, M., Ning, J., Du, Y., Cao, J., Zhang, D., Wang, J., \&
  Chen, M. (2020). Modelling the evolution trajectory of COVID-19 in Wuhan,
  China: Experience and suggestions. Public Health.

  \bibitem{peru1} Acuna-Zegarra, M. A., Comas-Garcia, A., Hernandez-Vargas,
  E., Santana-Cibrian, M., \& Velasco-Hernandez, J. X. (2020). The SARS-CoV-2
  epidemic outbreak: a review of plausible scenarios of containment and
  mitigation for Mexico. medRxiv.

  \bibitem{peru2} Acua-Zegarra, M. A., Santana-Cibrian, M., \&
  Velasco-Hernandez, J. X. (2020). Modeling behavioral change and COVID-19
  containment in Mexico: A trade-off between lockdown and compliance.
  Mathematical Biosciences, 108370.

 %   \bibitem{WhoDALY} Imai, N., Cori, A., Dorigatti, I., Baguelin, M., Donnelly, C.
 % A., Riley, S., \& Ferguson, N. M. (2020). Report 3: transmissibility of
 % 2019-nCov. Reference Source.


\end{thebibliography}
